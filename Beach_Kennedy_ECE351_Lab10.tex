%%%%%%%%%%%%%%%%%%%%%%%%%%%%%%%%%%%%%%%%%%%%%%%%%%%%%%%%%%%%%%%%
%                                                              %
% Kennedy Beach                                                %
% ECE 351-51                                                   %
% Lab 10                                                        %
% 4/5/2022                                                     %
%                                                              %
%                                                              %
%%%%%%%%%%%%%%%%%%%%%%%%%%%%%%%%%%%%%%%%%%%%%%%%%%%%%%%%%%%%%%%%
\documentclass[12pt]{report}
\usepackage[english]{babel}
%\usepackage{natbib}
\usepackage{url}
\usepackage[utf8x]{inputenc}
\usepackage{amsmath}
\usepackage{graphicx}
\graphicspath{{images/}}
\usepackage{parskip}
\usepackage{fancyhdr}
\usepackage{vmargin}
\usepackage{listings}
\usepackage{hyperref}
\usepackage{xcolor}
\usepackage{caption}
\usepackage{subcaption}
\usepackage{steinmetz}
\definecolor{codegreen}{rgb}{0,0.6,0}
\definecolor{codegray}{rgb}{0.5,0.5,0.5}
\definecolor{codeblue}{rgb}{0,0,0.95}
\definecolor{backcolour}{rgb}{0.95,0.95,0.92}
\lstdefinestyle{mystyle}{
backgroundcolor=\color{backcolour},
commentstyle=\color{codegreen},
keywordstyle=\color{codeblue},
numberstyle=\tiny\color{codegray},
stringstyle=\color{codegreen},
basicstyle=\ttfamily\footnotesize,
breakatwhitespace=false,
breaklines=true,
captionpos=b,
keepspaces=true,
numbers=left,
numbersep=5pt,
showspaces=false,
showstringspaces=false,
showtabs=false,
tabsize=2
}
\lstset{style=mystyle}
\setmarginsrb{3 cm}{2.5 cm}{3 cm}{2.5 cm}{1 cm}{1.5 cm}{1 cm}{1.5 cm}
\title{Lab 10}
% Title
\author{Kennedy Beach}
% Author
\date{April 5, 2022}
% Date
\makeatletter
\let\thetitle\@title
\let\theauthor\@author
\let\thedate\@date
\makeatother
\pagestyle{fancy}
\fancyhf{}
\rhead{\theauthor}
\lhead{\thetitle}
\cfoot{\thepage}
%%%%%%%%%%%%%%%%%%%%%%%%%%%%%%%%%%%%%%%%%%%%
\begin{document}
%%%%%%%%%%%%%%%%%%%%%%%%%%%%%%%%%%%%%%%%%%%%%%%%%%%%%%%%%%%%%%%%%%%%%%%%%%
%%%%%%%%%%%%%%%
\begin{titlepage}
\centering
\vspace*{0.5 cm}
% \includegraphics[scale = 0.075]{bsulogo.png}\\[1.0 cm] % 
\begin{center}    \textsc{\Large   ECE 351-51 }\\[2.0 cm]
\end{center}% University Name
\textsc{\Large April 5, 2022}\\[0.5 cm] % Course 
\rule{\linewidth}{0.2 mm} \\[0.4 cm]
{ \huge \bfseries \thetitle}\\
\rule{\linewidth}{0.2 mm} \\[1.5 cm]
\begin{minipage}{0.4\textwidth}
\begin{flushleft} \large
% \emph{Submitted To:}\\
% Name\\
% Affiliation\\
%contact info\\
\end{flushleft}
\end{minipage}~
\begin{minipage}{0.4\textwidth}
\begin{flushright} \large
\emph{Submitted By :} \\
Kennedy Beach
\end{flushright}
\end{minipage}\\[2 cm]
% \includegraphics[scale = 0.5]{PICMathLogo.png}
\end{titlepage}
%%%%%%%%%%%%%%%%%%%%%%%%%%%%%%%%%%%%%%%%%%%%%%%%%%%%%%%%%%%%%%%%%%%%%%%%%%
%%%%%%%%%%%%%%%
\tableofcontents
\pagebreak
%%%%%%%%%%%%%%%%%%%%%%%%%%%%%%%%%%%%%%%%%%%%%%%%%%%%%%%%%%%%%%%%%%%%%%%%%%
%%%%%%%%%%%%%%%
\renewcommand{\thesection}{\arabic{section}}
\section{Introduction}
The purpose of this lab was to use become familiar with frequency response tools and Bode plots using Python.
\section{Equations}

The transfer function, and the magnitude and phase of the frequency transfer function are shown below.
\begin{equation*}
H(s) = \frac{\frac{1}{RC}s}{s^2+\frac{1}{RC}s+\frac{1}{LC}}
\end{equation*}

\begin{equation*}
H(j\omega) = \frac{\frac{1}{RC}j\omega}{(j\omega)^2+\frac{1}{RC}j\omega+\frac{1}{LC}}
\end{equation*}

\begin{equation*}
|H(j\omega)| = \frac{\frac{\omega}{RC}}{\sqrt{(-\omega^2+\frac{1}{LC})^2+(\frac{\omega}{RC})^2}}
\end{equation*}

\begin{equation*}
\phase{H(j\omega)} = \frac{\pi}{2} - \arctan{(\frac{\frac{\omega}{RC}}{\frac{1}{LC}-\omega^2})}
\end{equation*}

The signal to be plotted and passed through the RLC circuit that the transfer function describes is seen below:
\begin{equation*}
x(t) = cos(2\pi*100t) + cos(2\pi*3024t) +sin(2\pi*50000t)
\end{equation*}

\section{Methodology}
The first part involved plotting the magnitude and phase of the transfer function that we found by hand. I did this through a user-defined function.

\begin{lstlisting}[language=Python, caption= User-defined function for the magnitude and phase]
import numpy as np
import matplotlib.pyplot as plt
import scipy.signal as sig
import control
import control as con

def H (omega, R, L, C):
    mag = (omega/(R*C)) / np.sqrt(( ((1/(L*C)) - (omega**2) )**2) + (omega/(R*C))**2)
    mag_dB = 20*np.log10(mag)
    phase_rad = (np.pi/2) - np.arctan(omega/(R*C) / (-omega**2 + 1/(L*C)))
    
    for i in range(len(phase_rad)):
        if phase_rad[i] >= np.pi/2:
            phase_rad[i] -= np.pi
        else:
            continue
        
    phase_deg = phase_rad*180/np.pi 
    
    return mag_dB, phase_deg

R = 1e3
L = 27e-3
C = 100e-9

steps = 10
omega = np.arange(1e3, 1e6, steps)

y1_mag, y1_phase = H(omega, R, L, C)
\end{lstlisting}

The Bode plot of the transfer function was then plotted with scipy.signal.bode(). The frequency response was also plotted using given code.

\begin{lstlisting}[language=Python, caption= scipy.signal.bode()]
num = [1/(R*C), 0]
den = [1, 1/(R*C), 1/(L*C)]

y2_freq, y2_mag, y2_phase = sig.bode((num, den), w=omega, n=steps)

sys = con.TransferFunction(num, den)
_ = con.bode(sys, omega, dB = True, Hz = True, deg = True, Plot = True)
\end{lstlisting}

The last part involved plotting the given signal, then passing it through the RLC filter using scipy.signal.bilinear() and scipy.signal.1filter().

\begin{lstlisting}[language=Python, caption= Filter Output]
fs = 50000
steps = 1/fs
t = np.arange(0, 0.01+steps, steps)
x = np.cos(2*np.pi*100*t) + np.cos(2*np.pi*3024*t) + np.sin(2*np.pi*50000*t)

plt.figure(figsize = (10, 7))
plt.ylabel('x(t)')
plt.xlabel('t')
plt.plot(t, x)
plt.grid()
plt.suptitle('Plot of x(t) where fs = 50000')

d_num, d_den = sig.bilinear(num, den, fs)
y_out = sig.lfilter(d_num, d_den, x)

plt.figure(figsize = (10, 7))
plt.ylabel('y(t)')
plt.xlabel('t')
plt.plot(t, y_out)
plt.grid()
plt.suptitle('Plot of y(t)')
\end{lstlisting}


\section{Results}

\begin{figure}[htp]
    \centering
    \includegraphics[width=16cm]{task1.png}
    \caption{Hand-calculated plot}
\end{figure}

\begin{figure}[htp]
    \centering
    \includegraphics[width=16cm]{task2.png}
    \caption{scipy.signal.bode()}
\end{figure}

\begin{figure}[htp]
    \centering
    \includegraphics[width=15cm]{task3.png}
    \caption{Bode Plot}
\end{figure}

\begin{figure}[htp]
    \centering
    \includegraphics[width=16cm]{task4.png}
    \caption{x(t)}
\end{figure}

\begin{figure}[htp]
    \centering
    \includegraphics[width=15cm]{task5.png}
    \caption{Output of x(t) through filter}
\end{figure}

\pagebreak
\section{Error Analysis}
There weren't any errors besides some simple mistypes.
\section{Questions}
\textbf{1. Explain how the filter and filtered output in Part 2 makes sense given the Bode plots from
Part 1. Discuss how the filter modifies specific frequency bands, in Hz.} \\ \\
Since the Bode plot descibed a bandpass filter, only certain frequencies are accepted while others are filtered out. This is seen in the output where the amplitude is mostly consistent with a uniform frequency.

\textbf{2. Discuss the purpose and workings of
scipy.signal.bilinear() and scipy.signal.lfilter().} \\ \\
The sig.bilinear function takes functions in the s-domain and transforms them into the z-domain. Once in the z-domain, the numerator and denominator were put into the sig.1filter function to run them through an electronic filter.

\textbf{3. What happens if you use a different sampling frequency in scipy.signal.bilinear() than
you used for the time-domain signal?} \\ \\
If I used a smaller frequency, the output would become more compact with waves that had smaller amplitudes. When I had a larger frequency, the output started to form a rough sine wave.

\textbf{4. Leave any feedback on the clarity of the expectations, instructions, and deliverables.} \\ \\
Everything was clear on what needed to be done and turned in.

\section{Conclusion}
This lab delved into Bode plots and outputs from a filter. The Python and \LaTeX{} code are seen in \url{https://github.com/Eniac618/ECE351_Code} and \url{https://github.com/Eniac618/ECE351_Reports} respectively.

\end{document}