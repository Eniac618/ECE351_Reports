%%%%%%%%%%%%%%%%%%%%%%%%%%%%%%%%%%%%%%%%%%%%%%%%%%%%%%%%%%%%%%%%
%                                                              %
% Kennedy Beach                                                %
% ECE 351-51                                                   %
% Lab 11                                                        %
% 4/12/2022                                                     %
%                                                              %
%                                                              %
%%%%%%%%%%%%%%%%%%%%%%%%%%%%%%%%%%%%%%%%%%%%%%%%%%%%%%%%%%%%%%%%
\documentclass[12pt]{report}
\usepackage[english]{babel}
%\usepackage{natbib}
\usepackage{url}
\usepackage[utf8x]{inputenc}
\usepackage{amsmath}
\usepackage{graphicx}
\graphicspath{{images/}}
\usepackage{parskip}
\usepackage{fancyhdr}
\usepackage{vmargin}
\usepackage{listings}
\usepackage{hyperref}
\usepackage{xcolor}
\usepackage{caption}
\usepackage{subcaption}
\usepackage{steinmetz}
\definecolor{codegreen}{rgb}{0,0.6,0}
\definecolor{codegray}{rgb}{0.5,0.5,0.5}
\definecolor{codeblue}{rgb}{0,0,0.95}
\definecolor{backcolour}{rgb}{0.95,0.95,0.92}
\lstdefinestyle{mystyle}{
backgroundcolor=\color{backcolour},
commentstyle=\color{codegreen},
keywordstyle=\color{codeblue},
numberstyle=\tiny\color{codegray},
stringstyle=\color{codegreen},
basicstyle=\ttfamily\footnotesize,
breakatwhitespace=false,
breaklines=true,
captionpos=b,
keepspaces=true,
numbers=left,
numbersep=5pt,
showspaces=false,
showstringspaces=false,
showtabs=false,
tabsize=2
}
\lstset{style=mystyle}
\setmarginsrb{3 cm}{2.5 cm}{3 cm}{2.5 cm}{1 cm}{1.5 cm}{1 cm}{1.5 cm}
\title{Lab 11}
% Title
\author{Kennedy Beach}
% Author
\date{April 12, 2022}
% Date
\makeatletter
\let\thetitle\@title
\let\theauthor\@author
\let\thedate\@date
\makeatother
\pagestyle{fancy}
\fancyhf{}
\rhead{\theauthor}
\lhead{\thetitle}
\cfoot{\thepage}
%%%%%%%%%%%%%%%%%%%%%%%%%%%%%%%%%%%%%%%%%%%%
\begin{document}
%%%%%%%%%%%%%%%%%%%%%%%%%%%%%%%%%%%%%%%%%%%%%%%%%%%%%%%%%%%%%%%%%%%%%%%%%%
%%%%%%%%%%%%%%%
\begin{titlepage}
\centering
\vspace*{0.5 cm}
% \includegraphics[scale = 0.075]{bsulogo.png}\\[1.0 cm] % 
\begin{center}    \textsc{\Large   ECE 351-51 }\\[2.0 cm]
\end{center}% University Name
\textsc{\Large April 12, 2022}\\[0.5 cm] % Course 
\rule{\linewidth}{0.2 mm} \\[0.4 cm]
{ \huge \bfseries \thetitle}\\
\rule{\linewidth}{0.2 mm} \\[1.5 cm]
\begin{minipage}{0.4\textwidth}
\begin{flushleft} \large
% \emph{Submitted To:}\\
% Name\\
% Affiliation\\
%contact info\\
\end{flushleft}
\end{minipage}~
\begin{minipage}{0.4\textwidth}
\begin{flushright} \large
\emph{Submitted By :} \\
Kennedy Beach
\end{flushright}
\end{minipage}\\[2 cm]
% \includegraphics[scale = 0.5]{PICMathLogo.png}
\end{titlepage}
%%%%%%%%%%%%%%%%%%%%%%%%%%%%%%%%%%%%%%%%%%%%%%%%%%%%%%%%%%%%%%%%%%%%%%%%%%
%%%%%%%%%%%%%%%
\tableofcontents
\pagebreak
%%%%%%%%%%%%%%%%%%%%%%%%%%%%%%%%%%%%%%%%%%%%%%%%%%%%%%%%%%%%%%%%%%%%%%%%%%
%%%%%%%%%%%%%%%
\renewcommand{\thesection}{\arabic{section}}
\section{Introduction}
The purpose of this lab was to use Python functions and Christopher Felton's function to analyze a discrete system.
\section{Equations}

The following causal function was provided:
\begin{equation*}
y[k] = 2x[k] - 40x[k-1] + 10y[k-1] - 16y[k-2]
\end{equation*}

The steps to find the transfer function are described below:
    \begin{align*}
        y[k] - 10[k-1] + 16y[k-2] &= 2x[k] - 40x[k-1] \\
        Y(z) - 10z^{-1}Y(z) + 16z^{-2} &= 2X(z) - 40z^{-1}X(z) \\
        Y(z)(1 - 10z^{-1} + 16z^{-2} &= X(z)(2 - 40z^{-1} \\
        \frac{Y(z)}{X(z)} &= \frac{2 - 40z^{-1}}{(1-10z^{-1}+16z^{-2})} \\
        H(z) &=\frac{2z(z-20)}{z^2 - 10z + 16} \\
    \end{align*}

The impulse response was found by using partial fraction expansion on the derived transfer function:
    \begin{align*}
        \frac{H(z)}{z} &= \frac{2(z-2)}{z^2 - 10z +16} \\
        \frac{H(z)}{z} &= \frac{A}{z-8} + \frac{B}{z-2} \\
        A &= \frac{2(z-20)}{z-8}|_{z=2} = 6 \\
        B &= \frac{2(z-20)}{z-2}|_{z=8} = -4 \\
        \frac{H(z)}{z} &= \frac{6}{z-8} - \frac{4}{z-2} \\
        Z^{-1}\{\frac{H(z)}{z}\} = h[k] &= [6(-8)^k - 4(-2)^k]u[k] \\
    \end{align*}


\section{Methodology}
The first part was to verify the partial fraction results using scipy.signal.residuez(). This is seen below and the output is seen in the Results section.

\begin{lstlisting}[language=Python, caption= scipy.signal.residuez()]
import numpy as np
import matplotlib.pyplot as plt
import scipy.signal as sig
import control
import control as con

num = [2, -40]
den = [1, -10, 16]

r, p, k = sig.residuez(num, den)

print("Task 3 Residue Results:\n r = {}\n p={}\n k = {}\n".format(r,p,k))
\end{lstlisting}

The Z-plane function provided by Felton was used to create the pole-zero plot for the transfer function.

\begin{lstlisting}[language=Python, caption= Z-plane function]
def zplane(b,a,filename=None):
    """Plot the complex z-plane given a transfer function.
    """
    import numpy as np
    import matplotlib.pyplot as plt
    from matplotlib import patches    
    
    # get a figure/plot
    ax = plt.subplot(111)
    # create the unit circle
    uc = patches.Circle((0,0), radius=1, fill=False,
                        color='black', ls='dashed')
    ax.add_patch(uc)
    # The coefficients are less than 1, normalize the coeficients
    if np.max(b) > 1:
        kn = np.max(b)
        b = np.array(b)/float(kn)
    else:
        kn = 1
    if np.max(a) > 1:
        kd = np.max(a)
        a = np.array(a)/float(kd)
    else:
        kd = 1
        
    # Get the poles and zeros
    p = np.roots(a)
    z = np.roots(b)
    k = kn/float(kd)
    
    # Plot the zeros and set marker properties    
    t1 = plt.plot(z.real, z.imag, 'o', ms=10,label='Zeros')
    plt.setp( t1, markersize=10.0, markeredgewidth=1.0)
    # Plot the poles and set marker properties
    t2 = plt.plot(p.real, p.imag, 'x', ms=10,label='Poles')
    plt.setp( t2, markersize=12.0, markeredgewidth=3.0)
    ax.spines['left'].set_position('center')
    ax.spines['bottom'].set_position('center')
    ax.spines['right'].set_visible(False)
    ax.spines['top'].set_visible(False)
    
    plt.legend()
    # set the ticks
    # r = 1.5; plt.axis('scaled'); plt.axis([-r, r, -r, r])
    # ticks = [-1, -.5, .5, 1]; plt.xticks(ticks); plt.yticks(ticks)
    if filename is None:
        plt.show()
    else:
        plt.savefig(filename)
    
    return z, p, k
    
z, p, k = zplane(num, den)
print('Zeros = ', z, '\nPoles = ',p)
\end{lstlisting}

The last part involved plotting the magnitude and phase responses of the transfer function using scipy.signal.freqz(). Since the output of scipy.signal.freqz() was in Hertz, I converted the magnitude of the output to decibels. 

\begin{lstlisting}[language=Python, caption= Magnitude and Phase Responses]
w, h = sig.freqz(num, den, whole=True)

plt.figure(figsize = (10, 7))
plt.subplot(2, 1, 1)
plt.ylabel('|H(jω)| (dB)')
plt.semilogx(w, 20*np.log10(np.abs(h)))
plt.grid()
plt.subtitle('Task 5 - Bode Plot of H(z) via sig.freqz() function')
plt.subplot (2, 1, 2)
plt.ylabel('/_H(jω)')
plt.semilogx(w, np.angle(h))
plt.grid()
\end{lstlisting}


\section{Results}

\begin{figure}[htp]
    \centering
    \includegraphics[width=12cm]{output.png}
    \caption{Residue Output}
\end{figure}

\begin{figure}[htp]
    \centering
    \includegraphics[width=12cm]{task4.png}
    \caption{Z-plane Function Output}
\end{figure}

\begin{figure}[htp]
    \centering
    \includegraphics[width=12cm]{task5.png}
    \caption{Bode Plot using scipy.signal.freqz()}
\end{figure}

\pagebreak
\section{Error Analysis}
There may be some potential errors with the bode plot since it is was not checked to make sure it is correct.
\section{Questions}
\textbf{1. Looking at the plot generated in Task 4, is H(z) stable? Explain why or why not.} \\ \\
H(z) is not stable since the poles and zeros are not within the unit circle.

\textbf{2. Leave any feedback on the clarity of the expectations, instructions, and deliverables.} \\ \\
Everything was clear on what needed to be done and turned in.

\section{Conclusion}
This lab delved into analyzing systems using the Z-domain and more of Python's capabilities. The Python and \LaTeX{} code are seen in \url{https://github.com/Eniac618/ECE351_Code} and \url{https://github.com/Eniac618/ECE351_Reports} respectively.

\end{document}