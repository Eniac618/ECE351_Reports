%%%%%%%%%%%%%%%%%%%%%%%%%%%%%%%%%%%%%%%%%%%%%%%%%%%%%%%%%%%%%%%%
%                                                              %
% Kennedy Beach                                                %
% ECE 351-51                                                   %
% Lab 4                                                        %
% 2/15/2022                                                     %
%                                                              %
%                                                              %
%%%%%%%%%%%%%%%%%%%%%%%%%%%%%%%%%%%%%%%%%%%%%%%%%%%%%%%%%%%%%%%%
\documentclass[12pt]{report}
\usepackage[english]{babel}
%\usepackage{natbib}
\usepackage{url}
\usepackage[utf8x]{inputenc}
\usepackage{amsmath}
\usepackage{graphicx}
\graphicspath{{images/}}
\usepackage{parskip}
\usepackage{fancyhdr}
\usepackage{vmargin}
\usepackage{listings}
\usepackage{hyperref}
\usepackage{xcolor}
\usepackage{caption}
\usepackage{subcaption}
\definecolor{codegreen}{rgb}{0,0.6,0}
\definecolor{codegray}{rgb}{0.5,0.5,0.5}
\definecolor{codeblue}{rgb}{0,0,0.95}
\definecolor{backcolour}{rgb}{0.95,0.95,0.92}
\lstdefinestyle{mystyle}{
backgroundcolor=\color{backcolour},
commentstyle=\color{codegreen},
keywordstyle=\color{codeblue},
numberstyle=\tiny\color{codegray},
stringstyle=\color{codegreen},
basicstyle=\ttfamily\footnotesize,
breakatwhitespace=false,
breaklines=true,
captionpos=b,
keepspaces=true,
numbers=left,
numbersep=5pt,
showspaces=false,
showstringspaces=false,
showtabs=false,
tabsize=2
}
\lstset{style=mystyle}
\setmarginsrb{3 cm}{2.5 cm}{3 cm}{2.5 cm}{1 cm}{1.5 cm}{1 cm}{1.5 cm}
\title{Lab 4}
% Title
\author{Kennedy Beach}
% Author
\date{February 15, 2022}
% Date
\makeatletter
\let\thetitle\@title
\let\theauthor\@author
\let\thedate\@date
\makeatother
\pagestyle{fancy}
\fancyhf{}
\rhead{\theauthor}
\lhead{\thetitle}
\cfoot{\thepage}
%%%%%%%%%%%%%%%%%%%%%%%%%%%%%%%%%%%%%%%%%%%%
\begin{document}
%%%%%%%%%%%%%%%%%%%%%%%%%%%%%%%%%%%%%%%%%%%%%%%%%%%%%%%%%%%%%%%%%%%%%%%%%%
%%%%%%%%%%%%%%%
\begin{titlepage}
\centering
\vspace*{0.5 cm}
% \includegraphics[scale = 0.075]{bsulogo.png}\\[1.0 cm] % 
\begin{center}    \textsc{\Large   ECE 351-51 }\\[2.0 cm]
\end{center}% University Name
\textsc{\Large February 15, 2022}\\[0.5 cm] % Course 
\rule{\linewidth}{0.2 mm} \\[0.4 cm]
{ \huge \bfseries \thetitle}\\
\rule{\linewidth}{0.2 mm} \\[1.5 cm]
\begin{minipage}{0.4\textwidth}
\begin{flushleft} \large
% \emph{Submitted To:}\\
% Name\\
% Affiliation\\
%contact info\\
\end{flushleft}
\end{minipage}~
\begin{minipage}{0.4\textwidth}
\begin{flushright} \large
\emph{Submitted By :} \\
Kennedy Beach
\end{flushright}
\end{minipage}\\[2 cm]
% \includegraphics[scale = 0.5]{PICMathLogo.png}
\end{titlepage}
%%%%%%%%%%%%%%%%%%%%%%%%%%%%%%%%%%%%%%%%%%%%%%%%%%%%%%%%%%%%%%%%%%%%%%%%%%
%%%%%%%%%%%%%%%
\tableofcontents
\pagebreak
%%%%%%%%%%%%%%%%%%%%%%%%%%%%%%%%%%%%%%%%%%%%%%%%%%%%%%%%%%%%%%%%%%%%%%%%%%
%%%%%%%%%%%%%%%
\renewcommand{\thesection}{\arabic{section}}
\section{Introduction}
The purpose of this lab was to convolve three functions using the user-defined convolution function from Lab 3 and then comparing it to hand calculations of the functions.

\section{Equations}
The following signal equations were used to create functions in Python:
\begin{equation*}
h1(t) = e^{-2t}*[u(t) - u(t-3)]
\end{equation*}
\begin{equation*}
h2(t) = u(t-2) - u(t-6)
\end{equation*}
\begin{equation*}
h3(t) = cos(w_0*t)*u(t) for f_0 = 0.25Hz
\end{equation*}

The convolutions of the transfer functions and the step function were calculated and the results are given below:
\begin{equation*}
    y1(t) = 0.5*[(1-e^{-2t})*u(t) - (1 - e^{-2(t-3)})*u(t-3)]
\end{equation*}
\begin{equation*}
    y2(t) = r(t-2)*u(t-2) - r(t-6)*u(t-6)
\end{equation*}
\begin{equation*}
    y3(t) = (1/w)*sin(wt)u(t)
\end{equation*}

\section{Methodology}
The first part involved plotting the three signals given above from -10 to 10. Listing 1 provides the necessary code for the user-defined functions as well as the ramp and step functions used previously. The plots are seen in the Results section.
\begin{lstlisting}[language=Python, caption=Defining the functions to plot the three signals from -10 \textless{} t \textless{} 10]
import numpy as np
import matplotlib.pyplot as plt
import math

def u(t):
    if t < 0:
        return 0
    if t >= 0:
        return 1
    
def r(t):
    if t < 0:
        return 0
    if t >= 0:
        return t
    
def h1(t):
    y = np.zeros((len(t), 1))
    for i in range(len(t)):
        y[i] = np.exp(-2*t[i])*(u(t[i])-u(t[i]-3))
    return y

def h2(t):
    y = np.zeros((len(t), 1))
    for i in range(len(t)):
        y[i] = u(t[i]-2) - u(t[i]-6)
    return y

def h3(t):
    f = 0.25
    w = f*2*np.pi
    y = np.zeros((len(t), 1))
    for i in range(len(t)):
        y[i] = math.cos(w*t[i]) * u(t[i])
    return y

steps = .01
t = np.arange(-10, 10 + steps, steps)

y1 = h1(t)
y2 = h2(t)
y3 = h3(t)
\end{lstlisting}

The user-defined convolution function is given below in Listing 2. This was used to convolve the transer functions with the step function. Another function was defined to create a step function array.

\begin{lstlisting}[language=Python, caption=Convolution and step array functions]
def my_conv(f1, f2):
    Nf1 = len(f1)           #variable with the length of f1
    Nf2 = len(f2)           #variable with the length of f2
    
    f1Ex = np.append(f1, np.zeros((1, Nf2-1)))  #creates an array that is the same size as f1 and f2
    f2Ex = np.append(f2, np.zeros((1, Nf1-1)))
    
    result = np.zeros(f1Ex.shape)   #creates a zero-filled array the same size as both functions
    
    for i in range((Nf2+Nf1-2)):    #goes through the length of f1 and f2
        result[i] = 0
        for j in range(Nf1):        #goes through the length of f1
            if (i-j+1 > 0):         #makes sure the loop doesn't go past 0 entries
                try: 
                    result[i] = result[i] + f1Ex[j]*f2Ex[i-j+1]     #combines the previous results with the product of the new entries
                except:
                    print(i,j)
    return result

#step function array
def u_array(t):
    y = np.zeros((len(t), 1))
    for i in range(len(t)):
        y[i] = u(t[i])
    return y

t = np.arange(-10, 10 + steps, steps)
time = len(t);
tEx = np.arange(-20, (2*t[time-1]) + steps, steps)

a = my_conv(y1, step)*steps
b = my_conv(y2, step)*steps
c = my_conv(y3, step)*steps
\end{lstlisting}

\begin{lstlisting}[language=Python, caption=Defining functions for plotting the hand calculations]
def c1(t):
    y = np.zeros((len(t),1))
    for i in range(len(t)):
        y[i] = 0.5*((1-np.exp(-2*t[i]))*u(t[i]) - ((1-np.exp(-2*(t[i]-3)))*u(t[i]-3)))
    return y

def c2(t):
    y = np.zeros((len(t),1))
    for i in range(len(t)):
        y[i] = r(t[i]-2)*u(t[i]-2) - r(t[i]-6)*u(t[i]-6)
    return y

def c3(t):
    f = 0.25
    w = 2*f*np.pi
    y = np.zeros((len(t),1))
    for i in range(len(t)):
        y[i] = (1/w)*math.sin(w*t[i])*u(t[i])
    return y
\end{lstlisting}

\section{Results}
The hand calculations foe the convolutions are shown below:
\begin{equation*}
    y1(t) = \int_0^t h1(T)u(t-T) dT
\end{equation*}
\begin{equation*}
        = \int_0^t e^{-2T}*[u(T) - u(T-3)]u(t-T) dT 
\end{equation*}
\begin{equation*}
    = 0.5[(1-e^{-2t})u(t) - (1-e^{-2(t-3)})u(t-3)]
\end{equation*}

\begin{equation*}
    y2(t) = \int_0^t h2(T)u(t-T) dT
\end{equation*}
\begin{equation*}
        = \int_0^t [u(T-2) - u(T-6)]u(t-T) dT 
\end{equation*}
\begin{equation*}
    = r(t-2)u(t-2) - r(t-6)u(t-6)
\end{equation*}

\begin{equation*}
    y3(t) = \int_0^t h3(T)u(t-T) dT
\end{equation*}
\begin{equation*}
        = \int_0^t cos(w_0*T)u(T)u(t-T) dT 
\end{equation*}
\begin{equation*}
    = (1/w_0)sin(w_0*t)u(t)
\end{equation*}

\begin{figure}[htp]
    \centering
    \includegraphics[width=15cm]{graphs.png}
    \caption{Plots of the three initial given signals}
\end{figure}

\begin{figure}[htp]
    \centering
    \includegraphics[width=15cm]{conv.png}
    \caption{User-defined convolutions of the three signals}
\end{figure}

\begin{figure}[htp]
    \centering
    \includegraphics[width=15cm]{hand.png}
    \caption{Convolutions using the scipy.signal library}
\end{figure}

\pagebreak
\section{Error Analysis}
My hand calculation convolution for the first transfer function doesn't look like the user-defined convolution. I tried reworking my equation, but I don't know if my equation itself is wrong, or if there is an error in the Python.
\section{Questions}
\textbf{1. Leave any feedback on the clarity of lab tasks, expectations, and deliverables.} \\ \\
Everything was clear on what needed to be done and turned in.

\section{Conclusion}
A comparison between hand calculated and Python convolution was explored. There was some error on the hand calculation side since the first convolution decreases at a sooner point than the user-defined convolution. There is also some scaling issues with the second hand convolution since compared to the user-defined one. The Python and \LaTeX{} code are seen in \url{https://github.com/Eniac618/ECE351_Code} and \url{https://github.com/Eniac618/ECE351_Reports} respectively.

\end{document}