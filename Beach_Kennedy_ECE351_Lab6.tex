%%%%%%%%%%%%%%%%%%%%%%%%%%%%%%%%%%%%%%%%%%%%%%%%%%%%%%%%%%%%%%%%
%                                                              %
% Kennedy Beach                                                %
% ECE 351-51                                                   %
% Lab 6                                                        %
% 3/1/2022                                                     %
%                                                              %
%                                                              %
%%%%%%%%%%%%%%%%%%%%%%%%%%%%%%%%%%%%%%%%%%%%%%%%%%%%%%%%%%%%%%%%
\documentclass[12pt]{report}
\usepackage[english]{babel}
%\usepackage{natbib}
\usepackage{url}
\usepackage[utf8x]{inputenc}
\usepackage{amsmath}
\usepackage{graphicx}
\graphicspath{{images/}}
\usepackage{parskip}
\usepackage{fancyhdr}
\usepackage{vmargin}
\usepackage{listings}
\usepackage{hyperref}
\usepackage{xcolor}
\usepackage{caption}
\usepackage{subcaption}
\definecolor{codegreen}{rgb}{0,0.6,0}
\definecolor{codegray}{rgb}{0.5,0.5,0.5}
\definecolor{codeblue}{rgb}{0,0,0.95}
\definecolor{backcolour}{rgb}{0.95,0.95,0.92}
\lstdefinestyle{mystyle}{
backgroundcolor=\color{backcolour},
commentstyle=\color{codegreen},
keywordstyle=\color{codeblue},
numberstyle=\tiny\color{codegray},
stringstyle=\color{codegreen},
basicstyle=\ttfamily\footnotesize,
breakatwhitespace=false,
breaklines=true,
captionpos=b,
keepspaces=true,
numbers=left,
numbersep=5pt,
showspaces=false,
showstringspaces=false,
showtabs=false,
tabsize=2
}
\lstset{style=mystyle}
\setmarginsrb{3 cm}{2.5 cm}{3 cm}{2.5 cm}{1 cm}{1.5 cm}{1 cm}{1.5 cm}
\title{Lab 6}
% Title
\author{Kennedy Beach}
% Author
\date{March 1, 2022}
% Date
\makeatletter
\let\thetitle\@title
\let\theauthor\@author
\let\thedate\@date
\makeatother
\pagestyle{fancy}
\fancyhf{}
\rhead{\theauthor}
\lhead{\thetitle}
\cfoot{\thepage}
%%%%%%%%%%%%%%%%%%%%%%%%%%%%%%%%%%%%%%%%%%%%
\begin{document}
%%%%%%%%%%%%%%%%%%%%%%%%%%%%%%%%%%%%%%%%%%%%%%%%%%%%%%%%%%%%%%%%%%%%%%%%%%
%%%%%%%%%%%%%%%
\begin{titlepage}
\centering
\vspace*{0.5 cm}
% \includegraphics[scale = 0.075]{bsulogo.png}\\[1.0 cm] % 
\begin{center}    \textsc{\Large   ECE 351-51 }\\[2.0 cm]
\end{center}% University Name
\textsc{\Large March 1, 2022}\\[0.5 cm] % Course 
\rule{\linewidth}{0.2 mm} \\[0.4 cm]
{ \huge \bfseries \thetitle}\\
\rule{\linewidth}{0.2 mm} \\[1.5 cm]
\begin{minipage}{0.4\textwidth}
\begin{flushleft} \large
% \emph{Submitted To:}\\
% Name\\
% Affiliation\\
%contact info\\
\end{flushleft}
\end{minipage}~
\begin{minipage}{0.4\textwidth}
\begin{flushright} \large
\emph{Submitted By :} \\
Kennedy Beach
\end{flushright}
\end{minipage}\\[2 cm]
% \includegraphics[scale = 0.5]{PICMathLogo.png}
\end{titlepage}
%%%%%%%%%%%%%%%%%%%%%%%%%%%%%%%%%%%%%%%%%%%%%%%%%%%%%%%%%%%%%%%%%%%%%%%%%%
%%%%%%%%%%%%%%%
\tableofcontents
\pagebreak
%%%%%%%%%%%%%%%%%%%%%%%%%%%%%%%%%%%%%%%%%%%%%%%%%%%%%%%%%%%%%%%%%%%%%%%%%%
%%%%%%%%%%%%%%%
\renewcommand{\thesection}{\arabic{section}}
\section{Introduction}
The purpose of this lab was to compare the hand-calculated and Python generated partial fraction expansions of some differential equations. A comparison of the step response generated by the cosine method, using the residue and poles determined by scipy.signal.residue(), and scipy.signal.step() was also performed.
\section{Equations}
The following equation was transformed into a transfer function:
\begin{equation*}
y^{''}(t) + 10y^{'}(t) + 24y(t) = x^{''}(t) + 6x^{'}(t) + 12x(t)
\end{equation*}

\begin{equation*}
    H(s) = \frac{s{^2}+6s+12}{s{^2}+10s+24}
\end{equation*}
From the transfer function, the step response was found:
\begin{equation*}
y(t) = (\frac{1}{2}-\frac{1}{2}e^{-4t}+e^{-6t})u(t)
\end{equation*}

The second equation analyzed is:
\begin{equation*}
y^{(5)}(t) + 18y^{(4)}(t) + 218y^{(3)}(t) + 2036y^{(2)}(t) + 25250y^{(1)}(t) = 25250x(t)
\end{equation*}

\section{Methodology}
The first part involved plotting the step response y(t) found in the prelab from 0 to 2 seconds. The transfer function found in the prelab was also plotted over the same time using scipy.signal.step().
\begin{lstlisting}[language=Python, caption=Step response and transfer function plots]
import numpy as np
import matplotlib.pyplot as plt
import scipy.signal as sig


def u(t):
    y = np.zeros(t.shape)
    
    for i in range(len(t)):
        if t[i] >= 0:
            y[i] = 1
        else:
            y[i] = 0
    return y

steps = 1e-5
t1 = np.arange(0, 2 + steps, steps)

num1 = [1, 6, 12]
den1 = [1, 10, 24]

y1 = (0.5-0.5*np.exp(-4*t1)+np.exp(-6*t1))*u(t1)
tout, yout = sig.step((num1, den1), T = t1)

\end{lstlisting}

Using scipy.signal.residue(), the residues, poles, and gain were output. The function used was the transer function multiplied by the Laplace transform of a step function which is 1/s.

\begin{lstlisting}[language=Python, caption=sig.residue() of the first differential equation]
num2 = [1, 6, 12]
den2 = [1, 10, 24, 0]

R1, P1, K1 = sig.residue(num2,den2)
\end{lstlisting}

Next was to use scipy.signal.residue() to find the residue, poles, and gain of the second equation differential equation.

\begin{lstlisting}[language=Python, caption=sig.residue() of the second differential equation]
num3 = [25250]
den3 = [1, 18, 218, 2036, 9085, 25250, 0]

R2, P2, K2 = sig.residue(num3,den3)
\end{lstlisting}

These results were then input into a function that performs the Cosine Method, and then that function was plotted from 0 to 4.5 seconds. A comparison using scipy.signal.step() was made.

\begin{lstlisting}[language=Python, caption=Cosine Method]
def cosine_method(R, P, t):
    y = np.zeros(t.shape)
    
    for i in range(len(R)):
        mag_k = np.abs(R[i])
        angle_k = np.angle(R[i])
        alpha = np.real(P[i])
        omega = np.imag(P[i])
        
        y = y + mag_k * np.exp(alpha*t) * np.cos(omega*t + angle_k) * u(t)
    return y
    
t2 = np.arange(0, 4.5 + steps, steps)
    
y2 = cosine_method(R2, P2, t2)

num4 = [25250]
den4 = [1, 18, 218, 2036, 9085, 25250]
tout2, yout2 = sig.step((num4, den4), T = t2)
\end{lstlisting}

\section{Results}

\begin{figure}[htp]
    \centering
    \includegraphics[width=16cm]{y1.png}
    \caption{Plots of the first differential equation}
\end{figure}

\begin{figure}[htp]
    \centering
    \includegraphics[width=16cm]{y2.png}
    \caption{Plots of the second differential equation}
\end{figure}

\begin{figure}[htp]
    \centering
    \includegraphics[width=16cm]{output.png}
    \caption{Printed output of the residue (R), poles (P), and gain (K)}
\end{figure}

\pagebreak
\section{Error Analysis}
I ran into an issue where the y of my cosine method function wouldn't return. I had to change my step function from what it was in previous labs in order for the function to return. An empty array with the length of t was defined, then a for loop went through each value of i to see if the step function would return a 1 or 0. Previously, an if/else statement was used where if t was equal to or greater than zero, 1 would be returned. Otherwise, a 0 would be returned.
\section{Questions}
\textbf{1. For a non-complex pole-residue term, you can still use the cosine method, explain why this
works.} \\ \\
If a pole-residue term is not complex, then omega and angle of k are zero. Since these terms are in the cosine function, the cosine would return a value of 1. The new equation would be $$ y_c(t) = |k|e^{\alpha t}u(t) $$

\textbf{2. Leave any feedback on the clarity of the expectations, instructions, and deliverables.} \\ \\
Everything was clear on what needed to be done and turned in.

\section{Conclusion}
I was a little confused about how to implement the cosine method until I realized that the numpy library has functions to get the real and imaginary parts of a result. Once I realized that, a new error where I couldn't return the value of the cosine method appeared. This was fixed by redoing my step function that I have used in previous labs. The Python and \LaTeX{} code are seen in \url{https://github.com/Eniac618/ECE351_Code} and \url{https://github.com/Eniac618/ECE351_Reports} respectively.

\end{document}