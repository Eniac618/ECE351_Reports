%%%%%%%%%%%%%%%%%%%%%%%%%%%%%%%%%%%%%%%%%%%%%%%%%%%%%%%%%%%%%%%%
%                                                              %
% Kennedy Beach                                                %
% ECE 351-51                                                   %
% Lab 8                                                        %
% 3/22/2022                                                     %
%                                                              %
%                                                              %
%%%%%%%%%%%%%%%%%%%%%%%%%%%%%%%%%%%%%%%%%%%%%%%%%%%%%%%%%%%%%%%%
\documentclass[12pt]{report}
\usepackage[english]{babel}
%\usepackage{natbib}
\usepackage{url}
\usepackage[utf8x]{inputenc}
\usepackage{amsmath}
\usepackage{graphicx}
\graphicspath{{images/}}
\usepackage{parskip}
\usepackage{fancyhdr}
\usepackage{vmargin}
\usepackage{listings}
\usepackage{hyperref}
\usepackage{xcolor}
\usepackage{caption}
\usepackage{subcaption}
\definecolor{codegreen}{rgb}{0,0.6,0}
\definecolor{codegray}{rgb}{0.5,0.5,0.5}
\definecolor{codeblue}{rgb}{0,0,0.95}
\definecolor{backcolour}{rgb}{0.95,0.95,0.92}
\lstdefinestyle{mystyle}{
backgroundcolor=\color{backcolour},
commentstyle=\color{codegreen},
keywordstyle=\color{codeblue},
numberstyle=\tiny\color{codegray},
stringstyle=\color{codegreen},
basicstyle=\ttfamily\footnotesize,
breakatwhitespace=false,
breaklines=true,
captionpos=b,
keepspaces=true,
numbers=left,
numbersep=5pt,
showspaces=false,
showstringspaces=false,
showtabs=false,
tabsize=2
}
\lstset{style=mystyle}
\setmarginsrb{3 cm}{2.5 cm}{3 cm}{2.5 cm}{1 cm}{1.5 cm}{1 cm}{1.5 cm}
\title{Lab 8}
% Title
\author{Kennedy Beach}
% Author
\date{March 22, 2022}
% Date
\makeatletter
\let\thetitle\@title
\let\theauthor\@author
\let\thedate\@date
\makeatother
\pagestyle{fancy}
\fancyhf{}
\rhead{\theauthor}
\lhead{\thetitle}
\cfoot{\thepage}
%%%%%%%%%%%%%%%%%%%%%%%%%%%%%%%%%%%%%%%%%%%%
\begin{document}
%%%%%%%%%%%%%%%%%%%%%%%%%%%%%%%%%%%%%%%%%%%%%%%%%%%%%%%%%%%%%%%%%%%%%%%%%%
%%%%%%%%%%%%%%%
\begin{titlepage}
\centering
\vspace*{0.5 cm}
% \includegraphics[scale = 0.075]{bsulogo.png}\\[1.0 cm] % 
\begin{center}    \textsc{\Large   ECE 351-51 }\\[2.0 cm]
\end{center}% University Name
\textsc{\Large March 22, 2022}\\[0.5 cm] % Course 
\rule{\linewidth}{0.2 mm} \\[0.4 cm]
{ \huge \bfseries \thetitle}\\
\rule{\linewidth}{0.2 mm} \\[1.5 cm]
\begin{minipage}{0.4\textwidth}
\begin{flushleft} \large
% \emph{Submitted To:}\\
% Name\\
% Affiliation\\
%contact info\\
\end{flushleft}
\end{minipage}~
\begin{minipage}{0.4\textwidth}
\begin{flushright} \large
\emph{Submitted By :} \\
Kennedy Beach
\end{flushright}
\end{minipage}\\[2 cm]
% \includegraphics[scale = 0.5]{PICMathLogo.png}
\end{titlepage}
%%%%%%%%%%%%%%%%%%%%%%%%%%%%%%%%%%%%%%%%%%%%%%%%%%%%%%%%%%%%%%%%%%%%%%%%%%
%%%%%%%%%%%%%%%
\tableofcontents
\pagebreak
%%%%%%%%%%%%%%%%%%%%%%%%%%%%%%%%%%%%%%%%%%%%%%%%%%%%%%%%%%%%%%%%%%%%%%%%%%
%%%%%%%%%%%%%%%
\renewcommand{\thesection}{\arabic{section}}
\section{Introduction}
The purpose of this lab was to hand-calculate the coefficient expressions of a Fourier series function, then use Python to display the series at different intervals.
\section{Equations}
\begin{figure}[htp]
    \centering
    \includegraphics[width=15cm]{plot.png}
    \caption{Given Square Wave}
\end{figure}

The coefficients and general expression of the Fourier series found are given below:
\begin{equation*}
a_k = 0
\end{equation*}

\begin{equation*}
b_k = \frac{2}{k\pi}(1-cos(k\pi))
\end{equation*}

\begin{equation*}
x(t) = \sum_{k=1}^{N} \frac{2}{k\pi}(1-cos(k\pi))*sin(k\frac{2\pi}{T}t)
\end{equation*}


\section{Methodology}
The first part involved defining the functions for the \(a_k\) and \(b_k\) coefficients. The equations are seen above.The values for \(a_0, a_1, b_1, b_2\), and \(b_3\) are seen in the output in the Results section.
\begin{lstlisting}[language=Python, caption=Coefficient functions]
import numpy as np
import matplotlib.pyplot as plt

def a_k(k):
    return 0;

def b_k(k):
    b_k = 2/(k*np.pi)*(1 - np.cos(k*np.pi))
    return b_k

\end{lstlisting}

The Fourier series expression, x(t), was then defined as a function.

\begin{lstlisting}[language=Python, caption=Fourier series function definition]
def fourier_series(N,T,t):
    omega = 2*np.pi/T
    y = np.zeros(t.shape)
    for i in range(1,N+1):
        y = y + (b_k(i)*np.sin(i*omega*t))
    return y
\end{lstlisting}

Lastly, the Fourier series was plotted for N = {1,3,15,50,150,1500} from 0 to 20 seconds, where T was 8 seconds.

\begin{lstlisting}[language=Python, caption=Steps and Fourier series variables that were plotted]
steps = 1e-5
t = np.arange(0, 20 + steps, steps)
x1 = fourier_series(1,8,t)
x3 = fourier_series(3,8,t)
x15 = fourier_series(15,8,t)
x50 = fourier_series(50,8,t)
x150 = fourier_series(150,8,t)
x1500 = fourier_series(1500,8,t)
\end{lstlisting}

\section{Results}

\begin{figure}[htp]
    \centering
    \includegraphics[width=16cm]{plot1_1.png}
    \caption{Plots of Fourier series where N = {1,3,15}}
\end{figure}

\begin{figure}[htp]
    \centering
    \includegraphics[width=16cm]{plot2_1.png}
    \caption{Plot of Fourier series where N = {50,150,1500}}
\end{figure}

\begin{figure}[htp]
    \centering
    \includegraphics[width=15cm]{output.png}
    \caption{Printed output of \(a_0, a_1, b_1, b_2\), and \(b_3\)}
\end{figure}

\pagebreak
\section{Error Analysis}
There was an initial issue where I had y in Fourier series function as an array and was adding the $i^{th}$ indexes. I had to add all of y together with the next $i^{th}$ \(b_k * sin(i*omega*t)\) term or else I would get "ValueError: setting an array element with a sequence".
\section{Questions}
\textbf{1. Is x(t) an even or an odd function? Explain why.} \\ \\
x(t) is an odd function since it is not symmetrical about the y-axis.

\textbf{2. Based on your results from Task 1, what do you expect the values of a2, a3, . . . , an to be?
Why?} \\ \\
The value of \(a_k\) is always 0 since that is what the equation simplified to. Also, since x(t) is an odd function, \(a_k\) will be zero since the integral of a cosine (sine) will be multiplied by a cosine which equates to 0.

\textbf{3. How does the approximation of the square wave change as the value of N increases? In what
way does the Fourier series struggle to approximate the square wave?} \\ \\
As N increases, the approximation gets closer to resembling a square wave. The Fourier series struggles to eliminate the vertical edges that go above the horizontal plateau of the square wave.

\textbf{4. What is occurring mathematically in the Fourier series summation as the value of N increases?} \\ \\
More summations of the terms will occur, which leads to a plot that looks more like a square wave. The individual summations get smaller and smaller as N, and k, increase.

\textbf{5. Leave any feedback on the clarity of the expectations, instructions, and deliverables.} \\ \\
Everything was clear on what needed to be done and turned in.

\section{Conclusion}
This lab was very simple since it simply showcased how increasing the number of summations in a Fourier series will approximate to a square wave. The Python and \LaTeX{} code are seen in \url{https://github.com/Eniac618/ECE351_Code} and \url{https://github.com/Eniac618/ECE351_Reports} respectively.

\end{document}